\documentclass[margin,line, 11pt]{res}

\topmargin=-0.5in    % Make letterhead start about 1 inch from top of page
\textheight=10in  % text height can be bigger for a longer letter
\oddsidemargin -.5in
\evensidemargin -.5in
\textwidth=6.5in
\itemsep=0in
\parsep=0in

\nofiles

% if using pdflatex:
\setlength{\pdfpagewidth}{\paperwidth}
\setlength{\pdfpageheight}{\paperheight}

\newenvironment{list1}{
  \begin{list}{\ding{113}}{%
      \setlength{\itemsep}{0in}
      \setlength{\parsep}{0in} \setlength{\parskip}{0in}
      \setlength{\topsep}{0in} \setlength{\partopsep}{0in}
      \setlength{\leftmargin}{0.17in}}}{\end{list}}
\newenvironment{list2}{
  \begin{list}{$\bullet$}{%
      \setlength{\itemsep}{0in}
      \setlength{\parsep}{0in} \setlength{\parskip}{0in}
      \setlength{\topsep}{0in} \setlength{\partopsep}{0in}
      \setlength{\leftmargin}{0.2in}}}{\end{list}}

\renewcommand{\familydefault}{\sfdefault}
%\usepackage[sfdefault, light, condensed]{roboto}
\usepackage[light,condensed]{iwona}
\usepackage[T1]{fontenc}
\usepackage[colorlinks=false,urlcolor=magenta,citecolor=blue,linkcolor=blue]{hyperref}

\begin{document}

\name{{\huge\bf Christopher H. Greer, Ph.D.}\vspace*{.05in}}

\begin{resume}
\vspace*{-3mm}
\section{Contact\\Information}
\begin{tabular}{@{}p{4.9in}p{4in}}
  (773) 680-6951 & \href{https://www.linkedin.com/in/chgreer/}{linkedin.com/in/chgreer/} \\
  \href{mailto:chgreer@gmail.com}{chgreer@gmail.com}  \\
\end{tabular}

% \vspace{-3mm}
% Collaborative, scientific thinker passionate about discovering and communicating nuanced insight from complicated data. Background includes: open-source contributions, project leadership, computer vision, traditional machine learning, and working with large, heterogeneous, often noisy datasets.
% \vspace*{-2mm}
\vspace*{-2mm}

\section{Professional \newline Experience}
\textbf{Children's Hospital Colorado} \hfill Aurora, Colorado\newline
\textit{Data Scientist Advanced} \hfill \textbf{November 2019 -- Present}\newline
I led implementation machine-learning models into real-time clinical practice. I developed new clinical models, provided software engineering and DevOps expertise, and piloted new technologies for clinical applications.
    \begin{list2}
%     	\vspace*{-5mm}
      \item Engineered a pipeline to serve predictions in live clinical records for: complications from influenza, pediatric septic shock, and serious bacterial infection. Served  $2000$ predictions per day with $99\%$ uptime and $\sim$minute latency. Responsible for $\sim5\%$ of all custom models running in Epic nationwide. (Epic, Python, Podman)
      \item Architected and built a Retrieval Augmented Generation (RAG) pattern with ChatGPT artificial intelligence to cohort patient using information contained in clinical notes. Increased precision relative to existing methods from 0.33 to 0.65 while maintaining 95\% recall. Eliminated 16 hours of manual review per 1000 patients. (Python, Azure OpenAI Service \& AI Search)
      \item Redesigned existing process to analyze employee survey comments using ChatGPT artificial intelligence in a RAG pattern, saving more than 300 developer hours per year. (Python, Azure OpenAI Service \& AI Search)
      % \item Leveraged discrete event simulation with a hospital digital twin to build census projections used during the COVID‑19 reponse and ongoing strategic planning. (Python, FutureFlow RX)
      \item Developed a risk stratification model for a hospital-acquired injury, attaining $80\%$ recall with $75\%$ precision on a highly-imbalanced ($2\%$ positive) dataset. Models were used to revamp nursing priorities as part of a hospital-wide performance improvement effort. (Python, R)
      \item Led data science team members modeling the risk of hospital acquired injury, likelihood of employee turnover, and a time series of hospital volumes using epidemiology data. Respiratory volume model contributed to earlier contracting for travel nurse resources, decreasing cost for the 2024 respiratory season. (Data Robot)
      % \item Designed a planning process to allocate resources and provide stakeholder visibility on commitments.
      \item Co-author on clinical trial results for the sepsis models (currently in JAMA peer review). Spoke nationally to audiences on implementation and MLOps pipelines for real-time EHR models.
    \end{list2}
\vspace*{-3mm}

\textbf{Oracle} \hfill Broomfield, Colorado\newline
\textit{Principal Data Scientist} \hfill \textbf{February 2017 -- November 2019}\newline
    \begin{list2}
    	\vspace*{-5mm}
      \item Incorporated geolocation data into the Oracle Identity Graph, saving $>\$1$ million in annual data costs. (Python, Spark, AWS EMR, Docker)
    	\item Designed and built a privacy-preserving record linkage algorithm, improving the quality of the match by 45\%, scale by 30\%, and standardizing the approach across 1000s of datasets. (Scala, Spark, EMR, Docker)
    	\item Designed and built a graph-quality measurement algorithm using a Monte-Carlo approach, demonstrating a $6\times$ improvement over deterministic graph approaches. (Scala, Spark, EMR, Docker)
    \end{list2}
\vspace*{-3mm}

\textbf{KPMG} \hfill Denver, Colorado\newline
\textit{Sr. Associate Data Scientist} \hfill \textbf{October 2015 -- February 2017}\newline
    \begin{list2}
    	\vspace*{-5mm}
      \item Designed and built a document classification tool for end-users. Created a domain-specific language for ease-of-use. (Apache OpenNLP, Spark, Python, Elasticsearch)
      % \item Built of suite of distributed tools to augment publicly available NLP packages.
      \item Utilized document classifier for information security and control for KPMG as well as data separation for large, multinational clients across millions of documents hundreds of TB in size.
    \end{list2}
%\vspace*{-2mm}

% \textbf{Steward Observatory, The University of Arizona} \hfill Tucson, Arizona\newline
% \textit{Postdoctoral Research Associate} \hfill \textbf{July, 2012 - September, 2015}\newline
%     \begin{list2}
%     	\vspace*{-5mm}
%       \item Breakthrough Prize in Fundamental Physics Laureate for contributions to the Event Horizon Telescope.
% %      \item Responsible for the receiver control and monitoring operating system using custom electronics (in C).
% %      \item Responsible for the optics design that integrated the receiver onto the telescope.
%     \end{list2}
% \vspace*{-2mm}
\vspace*{-2mm}

\section{Skills}
\textbf{Data:} Bayesian statistics, machine learning, natural language processing, Fourier signal analysis, MCMC, record linkage, visualization, large-language model (LLM) prompt engineering, LLM retrieval augmented generation (RAG)\\
\textbf{Technology:} Apache Spark, Python, MATLAB, C, SQL, git, BASH, Docker/Podman, Luigi, Azure DevOps, Epic electronic health record (active certifications in Cogito, Caboodle, and Cognitive Computing), Azure/AWS, Data Robot, Elasticsearch, R, Scala\\
% \textbf{Statistical Methods:} Hypothesis testing, error analysis, Monte Carlo methods, maximum likelihood\\ \\
\textbf{Leadership:} Experience organizing and leading workshops and collaboration meetings, supervising junior team members, public speaking, agile development, writing/publishing, 2020 Breakthrough Prize in Fundamental Physics laureate for contributions to the Event Horizon Telescope. \\
\vspace*{-2mm}

%section for two column education
\section{Education}
\begin{tabular}{@{}p{3in}p{3in}}
  \textbf{University of Chicago}, Chicago, IL
  \begin{list2}
  	\item Ph.D., Astronomy and Astrophysics, 2012
    \item M.S., Astronomy and Astrophysics, 2004
  \end{list2} &
  \textbf{Northwestern University}, Evanston, IL
  \begin{list2}
  	\item B.A., Physics and Mathematics, 2002
  \end{list2} \\
\end{tabular}
\vspace*{-4mm}

\end{resume}
\end{document}
