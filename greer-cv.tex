\documentclass[margin,line, 11pt]{res}

\topmargin=-0.5in    % Make letterhead start about 1 inch from top of page
\textheight=10in  % text height can be bigger for a longer letter
\oddsidemargin -.5in
\evensidemargin -.5in
\textwidth=6.5in
\itemsep=0in
\parsep=0in

\nofiles

% if using pdflatex:
\setlength{\pdfpagewidth}{\paperwidth}
\setlength{\pdfpageheight}{\paperheight}

\newenvironment{list1}{
  \begin{list}{\ding{113}}{%
      \setlength{\itemsep}{0in}
      \setlength{\parsep}{0in} \setlength{\parskip}{0in}
      \setlength{\topsep}{0in} \setlength{\partopsep}{0in}
      \setlength{\leftmargin}{0.17in}}}{\end{list}}
\newenvironment{list2}{
  \begin{list}{$\bullet$}{%
      \setlength{\itemsep}{0in}
      \setlength{\parsep}{0in} \setlength{\parskip}{0in}
      \setlength{\topsep}{0in} \setlength{\partopsep}{0in}
      \setlength{\leftmargin}{0.2in}}}{\end{list}}

\renewcommand{\familydefault}{\sfdefault}
%\usepackage[sfdefault, light, condensed]{roboto}
\usepackage[light,condensed]{iwona}
\usepackage[T1]{fontenc}
\usepackage[colorlinks=false,urlcolor=magenta,citecolor=blue,linkcolor=blue]{hyperref}

\begin{document}

\name{{\huge\bf Christopher H. Greer, Ph.D}\vspace*{.1in}}

\begin{resume}
\vspace*{-2mm}
\section{Basic\\Information}
\begin{tabular}{@{}p{4.9in}p{4in}}
  (773) 680-6951 & \href{https://www.linkedin.com/in/chgreer/}{linkedin.com/in/chgreer/} \\
  \href{mailto:chgreer@gmail.com}{chgreer@gmail.com}  \\
\end{tabular}

% \vspace{-3mm}
% Collaborative, scientific thinker passionate about discovering and communicating nuanced insight from complicated data. Background includes: open-source contributions, project leadership, computer vision, traditional machine learning, and working with large, heterogeneous, often noisy datasets.
% \vspace*{-2mm}

\section{Professional \newline Experience}
\textbf{Children's Hospital Colorado} \hfill Aurora, Colorado\newline
\textit{Data Scientist Advanced} \hfill \textbf{November, 2019 - Present}\newline
As the senior data scientist in the Analytics Resource Center, I lead implementation of real-time machine-learning models into the electronic health record (EHR) providing clinical decision support. I provide expertise around software engineering and devops best practices, and leverage new technologies for clinical applications (e.g. LLMs).
    \begin{list2}
%     	\vspace*{-5mm}
      \item Designed and built a pipeline for custom model implementation into the production EHR to predict: risk of complications from influenza, risk of septic shock, and risk of serious bacterial infection, (Epic, Python, Docker, Azure)
      \item Replaced existing NLP process to analyize employee survey comments using ChatGPT (Azure OpenAI Service) in a Retrieval Augmented Generation (RAG) pattern (Azure AI Search), saving of 400 developer hours per year. (Python, Azure OpenAI Service, AI Search)
      \item Designed and built a process using a RAG pattern for ChatGPT to build patient cohorts using information contained in clinical notes, doubling the precision of existing NLP techniques. Eliminated 16 hours of manual chart-review time per 1000 patients.  (Python, Azure OpenAI Service, AI Search)
      \item Leveraged discrete event simulation with a hospital digital twin to build census projections used during the COVID‑19 reponse and ongoing strategic planning. (Python, FutureFlow RX)
      \item Developed a risk stratification model for a serious hospital acquired injury attaining 80\% recall with 75\% precision; results to be used for allocating scarce nursing resources. (Python, R)
      \item Supervised junior team members to design and develop models predicting the risk of central-line associated bloodstream infections, likelihood of employee turnover, and a time series model of respiratory-season hospital volumes using epidemiology data. (Data Robot)
      \item Co-author on clinical trial results for the sepsis models. Spoke nationally to audiences on real-time implementation and MLOps pipelines for custom models in Epic.
    \end{list2}
\vspace*{-2mm}

\textbf{Oracle} \hfill Broomfield, Colorado\newline
\textit{Principle Data Scientist} \hfill \textbf{February, 2017 - November, 2019}\newline
    \begin{list2}
    	\vspace*{-5mm}
      \item Incorporated geolocation data into the Oracle Data Cloud Identity Graph (using Apache Spark on AWS EMR).
    	\item Rebuilt a privacy-preserving record linkage algorithm between incoming and fulfilled datasets, improving the quality of the match by 45\%, scale by 30\%, and standardizing the approach across 1000s of datasets.
    	\item Designed and built a graph-quality measurement algorithm using a Monte-Carlo approach, demonstrating a factor of $\sim 6$ improvement over deterministic graph approaches.
    \end{list2}
\vspace*{-2mm}

\textbf{KPMG} \hfill Denver, Colorado\newline
\textit{Sr. Associate Data Scientist} \hfill \textbf{October, 2015 - February, 2017}\newline
    \begin{list2}
    	\vspace*{-5mm}
      \item Architected a document classification tool for end-users. Wrote a domain-specific language for ease-of-use.  Used Apache OpenNLP, Spark, Python, and Elasticsearch.
      \item Built of suite of distributed tools to augment publicly available NLP packages.
      \item Used these tools for information security and control for KPMG as well as data separation for large, multinational clients across millions of documents hundreds of TB in size.
    \end{list2}
\vspace*{-2mm}

\textbf{Steward Observatory, The University of Arizona} \hfill Tucson, Arizona\newline
\textit{Postdoctoral Research Associate} \hfill \textbf{July, 2012 - September, 2015}\newline
    \begin{list2}
    	\vspace*{-5mm}
      \item 2020 Breakthrough Prize in Fundamental Physics Laureate for contributions to the Event Horizon Telescope project on the South Pole Telescope.
%      \item Responsible for the receiver control and monitoring operating system using custom electronics (in C).
%      \item Responsible for the optics design that integrated the receiver onto the telescope.
    \end{list2}
\vspace*{-2mm}

% \textbf{Dept. of Physics and Astronomy, Texas A\&M University} \hfill College Station, Texas\newline
% \textit{Ph.D Candidate} \hfill \textbf{August, 2010 - 2016}\newline
%     \begin{list2}
%     	\vspace*{-5mm}
%       \item Demonstrated that measurements from a planned large observation campaign could be improved by up to a factor of 3 over traditional statistical methods through the use of machine learning.
%       \item Implemented these machine learning methods and produced reliable results in a pilot survey of the real sky and under real-world conditions.
%     	% \item Collaborated with group members both in person, and through collaborative tools (e.g., GitHub, SVN).
%     	% \item Presented scientific results in high-impact, astrophysical journals and at international conferences.
%     \end{list2}
% \vspace*{-2mm}

% \textbf{The University of Tennessee}, Knoxville, Tennessee USA\newline
% \textit{Master's Candidate} \hfill \textbf{August, 2007 - 2009}\newline
%     \begin{list2}
%     	\vspace*{-5mm}
%       \item Implemented a C-based pipeline to process hundreds of GBs of simulation results. Including a computer vision algorithm to automatically analyze and compare results to expected targets.
%       \item Optimized simulation parameters using a genetic algorithm based search utilizing HPC (100k+ core) systems at the National Center for Computational Science, part of Oak Ridge National Laboratory
%     \end{list2}
% \vspace*{-3mm}

% \section{Awesome Projects}
% \textbf{Using Imaging to Infer Galaxy Properties}\newline
%     \begin{list2}
%     	\vspace*{-5mm}
%       \item Predicted galaxy chemical composition with $\sim$5\% error from pseudo-three color imaging, a result better than other current, similar efforts in the literature. Leveraged CNNs to analyze $\sim$150,000 images of galaxies.
%       \item Project start to publication: 4 months (typically $\sim$1.5 years). See: \href{https://github.com/boada/galaxy-cnns}{github.com/boada/galaxy-cnns}.
%     \end{list2}
%     \vspace*{-3mm}

% \textbf{Predicting Tournament Performance in Warmachine}\newline
%     \begin{list2}
%     	\vspace*{-5mm}
%     	\item Created an \href{https://en.wikipedia.org/wiki/Elo_rating_system}{Elo} based model to forecast the results of upcoming tournaments and identify potential upsets.
%     	\item Integrated predictions into a local community ranking system and forecasted $\sim$1800 tournament game results of the popular tabletop game using Python (e.g., Pandas).
%     \end{list2}
% \vspace*{-1mm}

\section{Skills}
\textbf{Data:} Bayesian statistics, machine learning, natural language processing, Fourier signal analysis, MCMC, record linkage, visualization, large-language model prompt engineering, retrieval augmented generation (RAG)\\
\textbf{Technology:} Docker, Luigi, Azure DevOps Pipelines, Apache Spark, Python, MATLAB, C, SQL, git, BASH, Epic electronic health record, Microsoft Azure, AWS, Data Robot, Elasticsearch, R, Scala\\
% \textbf{Statistical Methods:} Hypothesis testing, error analysis, Monte Carlo methods, maximum likelihood\\ \\
\textbf{Leadership:} Experience organizing and leading workshops and collaboration meetings, Teaching and mentoring junior team members, Public speaking, Agile development, writing/publishing. \\
\vspace*{-7mm}

%section for two column education
\section{Education}
\begin{tabular}{@{}p{3in}p{3in}}
  \textbf{University of Chicago}, Chicago, IL
  \begin{list2}
  	\item Ph.D., Astronomy and Astrophysics, 2012
    \item M.S., Astronomy and Astrophysics, 2004
  \end{list2} &
  \textbf{Northwestern University}, Evanston, IL
  \begin{list2}
  	\item B.A., Physics and Mathematics, 2002
  \end{list2} \\
\end{tabular}
\vspace*{-4mm}

\end{resume}
\end{document}
